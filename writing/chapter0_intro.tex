\chapter{Introduction}

With the advent of the third generation of DNA sequencing, new devices have emerged.
Particularly interesting is MinION, a small and portable platform for in-the-field use.
It utilizes a grid of nanopores which have an electric current flowing through them.
As the DNA strand is threaded through such nanopore, there are alternation in the
current which can be matched with a particular k-mer of DNA bases. Thanks to the
simplicity of such approach, long reads can be produced at a very fast pace. The
drawback however, is that the reads have a certain rate of error (around 10\% for the
newest generation).
Due to the errors, there's a need for the output data processing via probabilistic mo-
dels - in this instance, a Hidden Markov model. By feeding the obtained signals to such
model, we are able to mine useful information: for instance a probability distribution
for a set of DNA sequences that correspond best to the output from MinION.
As of now, only a small amount of information from the HMM is being used in
practice. For instance, we are able to find the most probable DNA sequence matched
to read when in fact taking more of the less probable sequences (which we will call
samples from now on) into consideration would yield more consistent and relevant
results. The idea of sampling is well known, however, there are currently no efficient
algorithms that could perform it on large data sets.
We attempt to speed up generating such samples by heuristics and using them for
alignment with a reference genome or other reads. Furthemore, we want to explore the
idea of speeding up the process of sampling by harnessing the paralelizaton power of
GPUs. Finally, the ultimate goal is to create a complete software tool combining all of
the above into one package that is user-friendly and open-source.